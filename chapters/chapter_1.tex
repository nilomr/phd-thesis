\section{Animal culture and social learning }
The idea that culture demarcated humans from other animals used to be widespread in Western academia. Over the past few decades this view was steadily challenged, and today it is common to find references to non-human animal cultures in scientific journals and the popular press alike (Whiten 2019a). To be sure, some energetically oppose the notion, and there is no shortage of disagreement over the definition of the term ‘culture’ (Laland \& Hoppitt 2003; Heyes 2020; kroeber defs of cult). But intricate and distinctive as human culture might be, the now burgeoning field of animal cultural research is showing us that the difference is one of degree and not kind (Whiten et al. 2017).

So, what do we mean by culture in this context? For our purposes, we can define it as any behavioural trait or information that is maintained in a population by virtue of being learnt from others---not genetically inherited, nor independently acquired. (See definitions in Whiten et al. 2017; Laland \& Hoppitt 2003.) Human ritual funerary practices are cultural; so are religions, the game of croquet, and PhD degrees. Crucially, under this definition, so are tool use in capuchin monkeys, homing efficiency in pigeons, the songs of many birds, feeding behaviours in humpback whales, and even mate preferences in fruit flies (Slater 2003; Allen et al. 2013; Falótico et al. 2019; Sasaki \& Biro 2017; Danchin et al. 2018). 

Social learning, where animals learn from observing or interacting with others, is widespread and a prerequisite for culture. While it may not always be advantageous (Henrich \& Boyd 1998; Giraldeau et al. 2002; Whitehead \& Richerson 2009), there is ample evidence that many of the skills that animals need to survive and reproduce can only be acquired by observing or interacting with others (Galef \& Laland 2005). Learning is more likely to occur from animals in close proximity or within the same social group, and this simple fact opens the door for behaviours to evolve differently in distinct populations, which can happen due to variation in learning abilities, ecological differences, or neutral processes (Araya-Salas et al. 2019; Mesoudi et al. 2016; Aplin 2016). When these differences, advantageous or not, accumulate and persist over time, cultural traditions emerge (Tchernichovski et al. 2017; Nunn et al. 2009). The resulting cultures can be transient or long-lasting, disorderly diverse or monolithically uniform: in two primate examples, chimpanzees (Pan troglodytes) may have used stone tools in a similar way for thousands of years (Mercader et al. 2007; Carvalho et al. 2008), while white-faced capuchin monkeys (Cebus capucinus) frequently invent and abandon quirky social conventions such as eyeball-poking, hand-sniffing and tail-sucking (Perry et al. 2003). 

\section{Cultural birds}
That animals’ lives have a cultural dimension was perhaps recognised earliest in birds. In 1920s South East England, some birds in the tit family started perforating the wax board or metal foil that sealed milk bottles to guzzle the cream accumulated at the top. This behaviour increased in frequency and geographic spread in the following decades, in what became a famous case of likely cultural transmission (Fisher \& Hinde 1949). Many years later, Aplin et al. (2015b) carried out experiments in a wild population of great tits \textit{Parus major} which demonstrated that new foraging behaviours can indeed spread socially and persist over more than one generation. Similarly, information acquired by individuals and groups of birds when flying along a route can accumulate in populations and, over time and even generations, lead to distinct migratory cultures (byholm2022, jesmer2018, berdahl2018, Sasaki \& Biro 2017).

While numerous examples of social learning and cultural phenomena exist in birds, it's their songs---thanks to their music-like qualities, and the remarkable ability of some species to imitate a wide range of sounds---that have garnered the most interest. Humanity’s captivation with bird song is far from a recent development, either. As far back as 350 B.C.E., in his work 'Historia Animalium,' Aristotle noted that birds, especially ‘broad-tongued’ ones, were capable of learning their songs; and that, sometimes, their voices changed with the ‘diversity of locality’ (Book 4, Chapter 9). Aristotle’s is one of the earliest recorded examples in a long tradition of analogies drawn between bird song and human language and music across times and cultures (Zirin 1980; Kleczkowska 2015).
Many centuries later, in 1650, the German Jesuit Athanasius Kircher would use musical notation to transcribe and analyse bird songs in his early musicology treatise, Musurgia Universalis, at a time when incorporating bird song-inspired phrases into instrumental compositions had become rather popular. However, it wouldn't be until after the Industrial Revolution that technological advancements enabled the first recordings of singing birds, a crucial step to study the changes and variations in their songs. (One earliest, if not the earliest, dates back to 1889, and was made by the pioneering sound recordist and broadcaster, Ludwig Koch, at the age of eight; british library2023.) Fast forward to the 1940s, and the invention of the sound spectrograph at the Bell Telephone Laboratories paved the way for a generation of researchers interested in bird song who were, for the first time, able to measure songs in unprecedented detail (Koenig et al. 1946; Baker 2001). An exhaustive study of the development of the song of the chaffinch Fringilla coelebs by Thorpe (1958) was followed by an explosion of interest in the matter, both in the field (Marler \& Tamura 1962, 1964) and the laboratory (see, for example, Nottebohm et al. 1976 on the neurobiology of song production), which continues unabated to this day. 







\section{A few notes to the reader}
Each chapter is an independent piece of work, which among other consequences means that the introduction to the study system, as well as some of the methods, are repeated in each chapter. If you want to save some time and redundancy, Chapter 3 §2 contains the most detailed description of the study system, fieldwork, and the data annotation process. Each chapter has a short preface where I have indicated the status of that particular manuscript and included any resources related to it, including websites and code repositories. I have spent a considerable amount of time trying to make sure that all the code and data (over 25k lines of code, much longer than this written thesis!) that I have produced during my PhD project are easily accessible and 

12391 pykanto
4549 demography
1634 greti hits web
4492 greti hits setup
2047 fieldtools
