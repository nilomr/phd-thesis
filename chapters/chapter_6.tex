
\section{Summary of the thesis}
\lettrine[lines=2]Bird song captivates human imagination. It has influenced music, literature, and inspired extensive research into its neurobiology, production, and function. Today, the study of bird song is experiencing a renaissance driven by new technological and theoretical tools. Advances in acoustic data collection, signal processing, and machine learning have made it feasible to collect and analyse vast datasets at lower costs. At the same time, insights from cultural evolution theory provide new conceptual tools that are allowing a richer understanding of diversity and variation in bird song, understood as a cultural phenomenon. This thesis tries to take advantage of both and contribute to our understanding of vocal behaviour and cultural dynamics within natural populations of birds. It makes contributions across four areas that are of significance for the research process (chapters 2 and 3) and bird song cultures (chapters 4 and 5).

When I started this project, I, very naively, in hindsight, thought that the methodological tools I would need were available. As I built enclosures for the audio recorders and planned the first field season, I began trying different workflows around existing tools---which was difficult given how little I knew about signal processing and programming---and nothing seemed to be quite right for the task, especially given just how much data I was going to try to collect. It was not planned at first, but as I went on, some of the tools and ideas that I was learning coalesced into a software library, which is now published and has already been used by other researchers. This is introduced in \autoref{chapter:2}. 

I knew that the kinds of questions about cultural change and diversity that I wanted to ask required large amounts of data, beyond what has typically been the case in fields such as behavioural ecology---it is simply difficult to collect lots of high-quality bird songs from individual birds in their natural habitats, so it is rarely done. During approximately nine months in the field, I, with the invaluable help of fantastic masters students Loanne Pichot in 2021 and Antoine Vansse in 2022, collected and later annotated over 100,000 songs from hundreds of birds. As part of this process, I was also lucky to contribute to the collective long-term monitoring effort in Wytham Woods, which involved ringing and measuring traits for around 2,000 great and blue tits over three years. Now, although my personal interests lie in cultural evolution, I know that there are many open research questions---about repeatability and stability, vocal individuality, and the connections between vocal performance, diversity, and fitness, to name a few---where research is hindered by the fact that individual-level datasets of songs from wild birds tend to be small and hard to come by. So, I made publishing all the song data and associated information on individual life histories a priority, hoping that others will benefit from my effort, too. \autoref{chapter:3} consists of a detailed description of the dataset and how I generated it.

Then, I use this extensive dataset to explore two processes, or rather, combinations of processes, which influence cultural change and diversity in animal groups:

The first of these concerns the demographic composition and structure of populations, which, although in a way extrinsic to culture, has been theorized to play a central role in cultural evolution. In \autoref{chapter:4}, I assess the strength of associations between variation in the demographic composition of neighbourhoods and cultural outcomes. I present evidence showing how individual-level differences in repertoire size and diversity can influence emergent group-level cultural dynamics. The results emphasize the need for empirical and modelling studies of cultural change to account for the demographic characteristics of populations, as well as their inherent heterogeneity across time and space: They shape individuals' learning opportunities, and this, in turn, can have large impacts on cultural diversity and turnover.

A second set of processes is intrinsic to the cognitive and motor abilities that support social learning and the formation of cultures. In the preceding chapter we saw that cultural turnover, the progressive replacement of song types, is very high at short temporal scales. And yet, this highly polymorphic culture is stable on larger spatial and temporal scales. In the final chapter, \autoref{chapter:5}, I use ideas from the study of musical and linguistic evolution and find preliminary evidence that, despite their diversity, great tit songs use fewer rhythmic and melodic patterns than would be expected by chance. These categories likely result from motor, information-processing, or perceptual biases, and, paralleling their role in human music, they might serve to stabilize cultural change---even in the face of high turnover.

\section{Future directions}
The week I began to get the hang of this and, for the first time, felt like I had learned enough to begin this research---that very same week I found myself facing the thesis submission deadline. That is to say that each of these chapters, and particularly the last two that present empirical findings, feels like an initial exploration. There is ample room for improvement, a few clear next steps, and some aspects of bird song culture that, while important, this thesis has not explored.

In \autoref{chapter:4}, I developed a method for re-identifying birds based on their songs. This method uses deep metric learning with a visual transformer model and has proven surprisingly effective: It enabled the identification of birds that could not be physically identified but were present in the vocal dataset across different years. Given its practical applications, I intend to conduct more thorough testing and create a more streamlined pipeline to make it accessible to other researchers.

\autoref{chapter:5} is still very much a work in progress. I started thinking about this only recently, but I think it holds some potential. I plan to use data from citizen science platforms to test if rhythmic and melodic structures are stable over very large spatial scales---if they can vary, this would represent evidence that cognitive biases can be socially acquired and evolve culturally---and to develop more formal tests of these ideas.

Although we compiled a large dataset containing both song recordings and individual-level fitness metrics (see \autoref{chapter:3}), I did not carry out any explicit research on the potential adaptive or selective consequences of cultural diversity or signal structure. Fitness is a complex trait influenced by many factors, such as environmental conditions, resource availability, predation pressure, territory quality, competition, and individual condition. Disentangling any specific fitness effects of cultural variation in songs from these confounding factors is extremely difficult, especially in a natural population where many variables cannot be controlled and there may be a bidirectional relationship between songs and individual fitness. Research attempting to do this often tests for correlations between a host of song characteristics and some fitness outcome, interpreting any significant results as evidence for direct selection: avoiding this problematic approach would require at the very least a full thesis of its own. So, instead, I chose to concentrate on investigating other, understudied processes shaping cultural diversity and turnover within the population, as well as the structure of the songs themselves. Why? Neutral or non-adaptive change is an important aspect of cultural evolution, and it happens to align most closely with my personal research interests. Be that as it may, exploring the potential adaptive significance of cultural diversity, turnover, and song structure---especially in the context of sexual selection---remains a logical next step for future research on this system.

The thesis presents evidence that demographic factors, such as population turnover, immigration, and age structure, can impact the diversity and pace of change in animal vocal cultures. However, cultural traits themselves may also influence demographic processes. For example, culturally transmitted migratory routes or feeding techniques may affect survival and reproductive success, shaping population growth and structure. Additionally, cultural preferences for certain mates or behaviours can influence patterns of dispersal and gene flow, further shaping demographic dynamics. Learning and song-sharing can also  reinforce or modifying existing social structures by affecting mate choice, territorial behaviour, or social hierarchies, which in turn can influence the transmission and maintenance of cultural traits. 

Similarly, cognitive and motor biases, such as perceptual or production constraints, are known to influence the structure and stability of cultural traits like bird songs. But, slightly annoyingly, cultural traits themselves could potentially shape or reinforce these biases. For example, which cultural variants are learned could influence the development and tuning of auditory perception or vocal motor skills in individuals while, at the same time, songs that align with existing biases may be more readily acquired and maintained, reinforcing or amplifying those biases over generations.

The potential for these recursive interactions between culture and the factors that shape it has important implications for our understanding of cultural evolution: it suggests that cultural processes cannot be studied in isolation but must be considered within a broader ecological and evolutionary context where culture may act as a source of evolutionary feedback, shaping the conditions that initially gave rise to it. Studying these feedback loops empirically will require conducting research over different timescales and using approaches different from those in this thesis, as well as integrating demographic, genetic, cognitive, and cultural data.

In this vein, an early and more humble goal of this thesis, which I never had time to fulfil (why does everything take so long?) was to integrate our knowledge about demographic processes in the Wytham great tit population with insights on structural bias in songs within the same modelling framework. This would allow us to estimate how they, together and separately, affect estimates of frequency-dependent learning biases and learning fidelity. Who knows what will come next, and if I will ever get around to doing any of this, but if you are reading this and are interested, please reach out.\\

\centering
\hspace{-.23cm}\adforn{26}
\centering




