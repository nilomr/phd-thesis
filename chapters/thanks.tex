While numerous examples of social learning and cultural phenomena exist in birds, it's their songs---thanks to their music-like qualities, and the remarkable ability of some species to imitate a wide range of sounds---that have garnered the most interest. Humanity’s captivation with bird song is far from a recent development, either. As far back as 350 B.C.E., in his work 'Historia Animalium,' Aristotle noted that birds, especially ‘broad-tongued’ ones, were capable of learning their songs; and that, sometimes, their voices changed with the ‘diversity of locality’ (Book 4, Chapter 9). Aristotle’s is one of the earliest recorded examples in a long tradition of analogies drawn between bird song and human language and music across times and cultures (Zirin 1980; Kleczkowska 2015).
Many centuries later, in 1650, the German Jesuit Athanasius Kircher would use musical notation to transcribe and analyse bird songs in his early musicology treatise, Musurgia Universalis, at a time when incorporating bird song-inspired phrases into instrumental compositions had become rather popular. However, it wouldn't be until after the Industrial Revolution that technological advancements enabled the first recordings of singing birds, a crucial step to study the changes and variations in their songs. (One earliest, if not the earliest, dates back to 1889, and was made by the pioneering sound recordist and broadcaster, Ludwig Koch, at the age of eight; british library2023.) Fast forward to the 1940s, and the invention of the sound spectrograph at the Bell Telephone Laboratories paved the way for a generation of researchers interested in bird song who were, for the first time, able to measure songs in unprecedented detail (Koenig et al. 1946; Baker 2001). An exhaustive study of the development of the song of the chaffinch Fringilla coelebs by Thorpe (1958) was followed by an explosion of interest in the matter, both in the field (Marler \& Tamura 1962, 1964) and the laboratory (see, for example, Nottebohm et al. 1976 on the neurobiology of song production), which continues unabated to this day. 