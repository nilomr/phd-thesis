% the abstract
The songs of some birds are learnt, much like human music and languages, and they change over time---they evolve culturally. Learning is imperfect and more likely to occur from nearby individuals, and this simple fact opens the door for behaviours to evolve differently in distinct populations. When these differences accumulate and persist over time, cultural traditions emerge. The resulting cultures can be transient or long-lasting, disorderly diverse, or monolithically uniform. This thesis addresses some of the challenges associated with studying cultural diversity in natural populations, and provides empirical evidence of factors that drive change and stability within vocal cultures. 

After a general introduction to bird song and cultural evolution in the first chapter, the second introduces Pykanto, an open-source Python library designed to process and help analyse large datasets of animal vocalizations recorded in the field. The third chapter presents a comprehensive and richly annotated open dataset of wild bird songs collected over three years from a single population of Great Tits (\textit{Parus major}) in the UK, which provides the data used in the remaining chapters. Then, the fourth chapter explores how demographic factors influence the frequency and diversity of cultural traits on small spatiotemporal scales, revealing that dispersal, immigration, and age structure are associated with variation in cultural diversity and turnover, and that this process is driven by individual-level differences in song repertoires. The fifth and final chapter characterises variation in the melodic and rhythmic structure of songs, suggesting that it is both non-random and categorically organised, which might contribute to large-scale stability in bird song cultures.

In summary, this thesis hopes to contribute to animal culture research by providing a substantial acoustic dataset of wild bird songs, introducing new computational tools for data analysis, and examining how extrinsic and intrinsic factors influence the diversity of bird song, rhythm, and melody; all of which are important for understanding emergent cultural processes at larger scales.
